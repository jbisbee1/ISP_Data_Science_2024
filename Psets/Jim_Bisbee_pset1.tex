% Options for packages loaded elsewhere
\PassOptionsToPackage{unicode}{hyperref}
\PassOptionsToPackage{hyphens}{url}
%
\documentclass[
]{article}
\usepackage{amsmath,amssymb}
\usepackage{iftex}
\ifPDFTeX
  \usepackage[T1]{fontenc}
  \usepackage[utf8]{inputenc}
  \usepackage{textcomp} % provide euro and other symbols
\else % if luatex or xetex
  \usepackage{unicode-math} % this also loads fontspec
  \defaultfontfeatures{Scale=MatchLowercase}
  \defaultfontfeatures[\rmfamily]{Ligatures=TeX,Scale=1}
\fi
\usepackage{lmodern}
\ifPDFTeX\else
  % xetex/luatex font selection
\fi
% Use upquote if available, for straight quotes in verbatim environments
\IfFileExists{upquote.sty}{\usepackage{upquote}}{}
\IfFileExists{microtype.sty}{% use microtype if available
  \usepackage[]{microtype}
  \UseMicrotypeSet[protrusion]{basicmath} % disable protrusion for tt fonts
}{}
\makeatletter
\@ifundefined{KOMAClassName}{% if non-KOMA class
  \IfFileExists{parskip.sty}{%
    \usepackage{parskip}
  }{% else
    \setlength{\parindent}{0pt}
    \setlength{\parskip}{6pt plus 2pt minus 1pt}}
}{% if KOMA class
  \KOMAoptions{parskip=half}}
\makeatother
\usepackage{xcolor}
\usepackage[margin=1in]{geometry}
\usepackage{color}
\usepackage{fancyvrb}
\newcommand{\VerbBar}{|}
\newcommand{\VERB}{\Verb[commandchars=\\\{\}]}
\DefineVerbatimEnvironment{Highlighting}{Verbatim}{commandchars=\\\{\}}
% Add ',fontsize=\small' for more characters per line
\usepackage{framed}
\definecolor{shadecolor}{RGB}{248,248,248}
\newenvironment{Shaded}{\begin{snugshade}}{\end{snugshade}}
\newcommand{\AlertTok}[1]{\textcolor[rgb]{0.94,0.16,0.16}{#1}}
\newcommand{\AnnotationTok}[1]{\textcolor[rgb]{0.56,0.35,0.01}{\textbf{\textit{#1}}}}
\newcommand{\AttributeTok}[1]{\textcolor[rgb]{0.13,0.29,0.53}{#1}}
\newcommand{\BaseNTok}[1]{\textcolor[rgb]{0.00,0.00,0.81}{#1}}
\newcommand{\BuiltInTok}[1]{#1}
\newcommand{\CharTok}[1]{\textcolor[rgb]{0.31,0.60,0.02}{#1}}
\newcommand{\CommentTok}[1]{\textcolor[rgb]{0.56,0.35,0.01}{\textit{#1}}}
\newcommand{\CommentVarTok}[1]{\textcolor[rgb]{0.56,0.35,0.01}{\textbf{\textit{#1}}}}
\newcommand{\ConstantTok}[1]{\textcolor[rgb]{0.56,0.35,0.01}{#1}}
\newcommand{\ControlFlowTok}[1]{\textcolor[rgb]{0.13,0.29,0.53}{\textbf{#1}}}
\newcommand{\DataTypeTok}[1]{\textcolor[rgb]{0.13,0.29,0.53}{#1}}
\newcommand{\DecValTok}[1]{\textcolor[rgb]{0.00,0.00,0.81}{#1}}
\newcommand{\DocumentationTok}[1]{\textcolor[rgb]{0.56,0.35,0.01}{\textbf{\textit{#1}}}}
\newcommand{\ErrorTok}[1]{\textcolor[rgb]{0.64,0.00,0.00}{\textbf{#1}}}
\newcommand{\ExtensionTok}[1]{#1}
\newcommand{\FloatTok}[1]{\textcolor[rgb]{0.00,0.00,0.81}{#1}}
\newcommand{\FunctionTok}[1]{\textcolor[rgb]{0.13,0.29,0.53}{\textbf{#1}}}
\newcommand{\ImportTok}[1]{#1}
\newcommand{\InformationTok}[1]{\textcolor[rgb]{0.56,0.35,0.01}{\textbf{\textit{#1}}}}
\newcommand{\KeywordTok}[1]{\textcolor[rgb]{0.13,0.29,0.53}{\textbf{#1}}}
\newcommand{\NormalTok}[1]{#1}
\newcommand{\OperatorTok}[1]{\textcolor[rgb]{0.81,0.36,0.00}{\textbf{#1}}}
\newcommand{\OtherTok}[1]{\textcolor[rgb]{0.56,0.35,0.01}{#1}}
\newcommand{\PreprocessorTok}[1]{\textcolor[rgb]{0.56,0.35,0.01}{\textit{#1}}}
\newcommand{\RegionMarkerTok}[1]{#1}
\newcommand{\SpecialCharTok}[1]{\textcolor[rgb]{0.81,0.36,0.00}{\textbf{#1}}}
\newcommand{\SpecialStringTok}[1]{\textcolor[rgb]{0.31,0.60,0.02}{#1}}
\newcommand{\StringTok}[1]{\textcolor[rgb]{0.31,0.60,0.02}{#1}}
\newcommand{\VariableTok}[1]{\textcolor[rgb]{0.00,0.00,0.00}{#1}}
\newcommand{\VerbatimStringTok}[1]{\textcolor[rgb]{0.31,0.60,0.02}{#1}}
\newcommand{\WarningTok}[1]{\textcolor[rgb]{0.56,0.35,0.01}{\textbf{\textit{#1}}}}
\usepackage{graphicx}
\makeatletter
\def\maxwidth{\ifdim\Gin@nat@width>\linewidth\linewidth\else\Gin@nat@width\fi}
\def\maxheight{\ifdim\Gin@nat@height>\textheight\textheight\else\Gin@nat@height\fi}
\makeatother
% Scale images if necessary, so that they will not overflow the page
% margins by default, and it is still possible to overwrite the defaults
% using explicit options in \includegraphics[width, height, ...]{}
\setkeys{Gin}{width=\maxwidth,height=\maxheight,keepaspectratio}
% Set default figure placement to htbp
\makeatletter
\def\fps@figure{htbp}
\makeatother
\setlength{\emergencystretch}{3em} % prevent overfull lines
\providecommand{\tightlist}{%
  \setlength{\itemsep}{0pt}\setlength{\parskip}{0pt}}
\setcounter{secnumdepth}{-\maxdimen} % remove section numbering
\ifLuaTeX
  \usepackage{selnolig}  % disable illegal ligatures
\fi
\usepackage{bookmark}
\IfFileExists{xurl.sty}{\usepackage{xurl}}{} % add URL line breaks if available
\urlstyle{same}
\hypersetup{
  pdftitle={Problem Set 1},
  pdfauthor={Jim Bisbee},
  hidelinks,
  pdfcreator={LaTeX via pandoc}}

\title{Problem Set 1}
\usepackage{etoolbox}
\makeatletter
\providecommand{\subtitle}[1]{% add subtitle to \maketitle
  \apptocmd{\@title}{\par {\large #1 \par}}{}{}
}
\makeatother
\subtitle{Intro to \texttt{R}}
\author{Jim Bisbee}
\date{Due Date: 2024-07-04}

\begin{document}
\maketitle

\subsection{Getting Set Up}\label{getting-set-up}

Open \texttt{RStudio} and create a new RMarkDown file (\texttt{.Rmd}) by
going to
\texttt{File\ -\textgreater{}\ New\ File\ -\textgreater{}\ R\ Markdown...}.
Accept defaults and save this file as \texttt{{[}LAST\ NAME{]}\_ps1.Rmd}
to your \texttt{code} folder.

Copy and paste the contents of this \texttt{.Rmd} file into your
\texttt{{[}LAST\ NAME{]}\_ps1.Rmd} file. Then change the
\texttt{author:\ {[}Your\ Name{]}} to your name.

We will be using the \texttt{sc\_debt.Rds} file from the course
\href{https://github.com/jbisbee1/ISP_Data_Science_2024/blob/main/data/sc_debt.Rds}{github
page}.

All of the following questions should be answered in this \texttt{.Rmd}
file. There are code chunks with incomplete code that need to be filled
in.

This problem set is worth 10 total points, plus two extra credit points.
The point values for each question are indicated in brackets below. To
receive full credit, you must have the correct code. In addition, some
questions ask you to provide a written response in addition to the code.

You are free to rely on whatever resources you need to complete this
problem set, including lecture notes, lecture presentations, Google,
your classmates\ldots you name it. However, the final submission must be
complete by you. There are no group assignments. To submit, email the
knitted output to Eun Ji Kim
(\href{mailto:kej990804@snu.ac.kr}{\nolinkurl{kej990804@snu.ac.kr}})
\textbf{as a PDF} by the start of class on Thursday, July 4th. If you
need help converting to a PDF, see
\href{https://github.com/jbisbee1/ISP_Data_Science_2024/blob/main/Psets/ISP_pset_0_HELPER.pdf}{this
tutorial}.

\textbf{Good luck!}

*Copy the link to ChatGPT you used here:
\_\_\_\_\_\_\_\_\_\_\_\_\_\_\_\_\_

\subsection{Question 0 {[}0 points{]}}\label{question-0-0-points}

\emph{Require \texttt{tidyverse} and load the \texttt{sc\_debt.Rds} data
by assigning it to an object named \texttt{df}.}

\begin{Shaded}
\begin{Highlighting}[]
\FunctionTok{require}\NormalTok{(tidyverse) }\CommentTok{\# Load tidyverse}
\end{Highlighting}
\end{Shaded}

\begin{verbatim}
## Loading required package: tidyverse
\end{verbatim}

\begin{verbatim}
## -- Attaching core tidyverse packages ------------------------ tidyverse 2.0.0 --
## v dplyr     1.1.4     v readr     2.1.5
## v forcats   1.0.0     v stringr   1.5.1
## v ggplot2   3.5.1     v tibble    3.2.1
## v lubridate 1.9.3     v tidyr     1.3.1
## v purrr     1.0.2     
## -- Conflicts ------------------------------------------ tidyverse_conflicts() --
## x dplyr::filter() masks stats::filter()
## x dplyr::lag()    masks stats::lag()
## i Use the conflicted package (<http://conflicted.r-lib.org/>) to force all conflicts to become errors
\end{verbatim}

\begin{Shaded}
\begin{Highlighting}[]
\NormalTok{df }\OtherTok{\textless{}{-}} \FunctionTok{read\_rds}\NormalTok{(}\StringTok{"https://github.com/jbisbee1/ISP\_Data\_Science\_2024/raw/main/data/sc\_debt.Rds"}\NormalTok{) }\CommentTok{\# Load the dataset directly from github}
\end{Highlighting}
\end{Shaded}

\subsection{Question 1 {[}1 point{]}}\label{question-1-1-point}

\emph{Which school has the lowest admission rate (\texttt{adm\_rate})
and which state is it in (\texttt{stabbr})?}

\begin{Shaded}
\begin{Highlighting}[]
\NormalTok{df }\SpecialCharTok{\%\textgreater{}\%} 
  \FunctionTok{arrange}\NormalTok{(adm\_rate) }\SpecialCharTok{\%\textgreater{}\%} \CommentTok{\# Arrange by the admission rate}
  \FunctionTok{select}\NormalTok{(instnm,adm\_rate,stabbr) }\CommentTok{\# Select the school name, the admission rate, and the state}
\end{Highlighting}
\end{Shaded}

\begin{verbatim}
## # A tibble: 2,546 x 3
##    instnm                                      adm_rate stabbr
##    <chr>                                          <dbl> <chr> 
##  1 Saint Elizabeth College of Nursing            0      NY    
##  2 Yeshivat Hechal Shemuel                       0      NY    
##  3 Hampshire College                             0.0197 MA    
##  4 Curtis Institute of Music                     0.0393 PA    
##  5 Stanford University                           0.0434 CA    
##  6 Harvard University                            0.0464 MA    
##  7 Pacific Oaks College                          0.0511 CA    
##  8 Columbia University in the City of New York   0.0545 NY    
##  9 Princeton University                          0.0578 NJ    
## 10 Yale University                               0.0608 CT    
## # i 2,536 more rows
\end{verbatim}

\begin{quote}
There are two schools: Saint Elizabeth College of Nursing and Yeshivat
Hechal Shemuel. Both have admissions rates of 0, meaning they don't
accept anyone and both are located in NY.
\end{quote}

\subsection{Question 2 {[}1 point{]}}\label{question-2-1-point}

\emph{Which are the top 10 schools by average SAT score
(\texttt{sat\_avg})?}

\begin{Shaded}
\begin{Highlighting}[]
\NormalTok{df }\SpecialCharTok{\%\textgreater{}\%}
  \FunctionTok{arrange}\NormalTok{(}\SpecialCharTok{{-}}\NormalTok{sat\_avg) }\SpecialCharTok{\%\textgreater{}\%} \CommentTok{\# arrange by SAT scores in descending order}
  \FunctionTok{select}\NormalTok{(instnm,sat\_avg) }\SpecialCharTok{\%\textgreater{}\%} \CommentTok{\# Select the school name and SAT score}
  \FunctionTok{print}\NormalTok{(}\AttributeTok{n =} \DecValTok{12}\NormalTok{) }\CommentTok{\# Print the first 12 rows (hint: there is a tie)}
\end{Highlighting}
\end{Shaded}

\begin{verbatim}
## # A tibble: 2,546 x 2
##    instnm                                 sat_avg
##    <chr>                                    <int>
##  1 California Institute of Technology        1557
##  2 Massachusetts Institute of Technology     1547
##  3 University of Chicago                     1528
##  4 Harvey Mudd College                       1526
##  5 Duke University                           1522
##  6 Franklin W Olin College of Engineering    1522
##  7 Washington University in St Louis         1520
##  8 Rice University                           1520
##  9 Yale University                           1517
## 10 Harvard University                        1517
## 11 Princeton University                      1517
## 12 Vanderbilt University                     1515
## # i 2,534 more rows
\end{verbatim}

\begin{quote}
Here are the top 10 schools by average SAT score.
\end{quote}

\subsection{Question 3 {[}1 point{]}}\label{question-3-1-point}

\emph{Create a new variable called \texttt{adm\_rate\_pct} which is the
admissions rate multiplied by 100 to convert from a 0-to-1 decimal to a
0-to-100 percentage point.}

\begin{Shaded}
\begin{Highlighting}[]
\NormalTok{df }\OtherTok{\textless{}{-}}\NormalTok{ df }\SpecialCharTok{\%\textgreater{}\%} \CommentTok{\# Use the object assignment operator to overwrite the df object}
  \FunctionTok{mutate}\NormalTok{(}\AttributeTok{adm\_rate\_pct =}\NormalTok{ adm\_rate}\SpecialCharTok{*}\DecValTok{100}\NormalTok{) }\CommentTok{\# Create the new variable adm\_rate\_pct}
\end{Highlighting}
\end{Shaded}

\subsection{Question 4 {[}1 point{]}}\label{question-4-1-point}

\emph{Calculate the average SAT score and median earnings of recent
graduates by state.}

\begin{Shaded}
\begin{Highlighting}[]
\NormalTok{df }\SpecialCharTok{\%\textgreater{}\%}
  \FunctionTok{group\_by}\NormalTok{(stabbr) }\SpecialCharTok{\%\textgreater{}\%} \CommentTok{\# Calculate state{-}by{-}state with group\_by()}
  \FunctionTok{summarise}\NormalTok{(}\AttributeTok{sat\_avg =} \FunctionTok{mean}\NormalTok{(sat\_avg,}\AttributeTok{na.rm=}\NormalTok{T), }\CommentTok{\# Summarise the average SAT}
            \AttributeTok{earn\_avg =} \FunctionTok{mean}\NormalTok{(md\_earn\_wne\_p6,}\AttributeTok{na.rm=}\NormalTok{T)) }\CommentTok{\# Summarise the average earnings}
\end{Highlighting}
\end{Shaded}

\begin{verbatim}
## # A tibble: 51 x 3
##    stabbr sat_avg earn_avg
##    <chr>    <dbl>    <dbl>
##  1 AK       1121    33300 
##  2 AL       1123.   28082.
##  3 AR       1141.   30452.
##  4 AZ       1147.   27613.
##  5 CA       1183.   33017.
##  6 CO       1132.   33955.
##  7 CT       1194.   35994.
##  8 DC       1262    41325 
##  9 DE       1043    32443.
## 10 FL       1142.   30318.
## # i 41 more rows
\end{verbatim}

\subsection{Question 5 {[}1 points{]}}\label{question-5-1-points}

\emph{Research Question: Do students who graduate from smaller schools
(i.e., schools with smaller student bodies) make more money in their
future careers? Before looking at the data, write out what you think the
answer is, and explain why you think so.}

\begin{quote}
Write a few sentences here.
\end{quote}

\subsection{Question 6 {[}2 points{]}}\label{question-6-2-points}

\emph{Based on this research question, what is the outcome / dependent /
\(Y\) variable and what is the explanatory / independent / \(X\)
variable? Create the scatterplot of the data based on this answer, along
with a line of best fit. Is your answer to the research question
supported?}

\begin{Shaded}
\begin{Highlighting}[]
\NormalTok{df }\SpecialCharTok{\%\textgreater{}\%}
  \FunctionTok{ggplot}\NormalTok{(}\FunctionTok{aes}\NormalTok{(}\AttributeTok{x =}\NormalTok{ , }\CommentTok{\# Put the explanatory variable on the x{-}axis}
             \AttributeTok{y =}\NormalTok{ )) }\SpecialCharTok{+}  \CommentTok{\# Put the outcome variable on the y{-}axis}
  \FunctionTok{geom\_point}\NormalTok{() }\SpecialCharTok{+} \CommentTok{\# Create a scatterplot}
  \FunctionTok{geom\_smooth}\NormalTok{() }\SpecialCharTok{+} \CommentTok{\# Add line of best fit}
  \FunctionTok{labs}\NormalTok{(}\AttributeTok{title =} \StringTok{\textquotesingle{}\textquotesingle{}}\NormalTok{, }\CommentTok{\# give the plot meaningful labels to help the viewer understand it}
       \AttributeTok{x =} \StringTok{\textquotesingle{}\textquotesingle{}}\NormalTok{,}
       \AttributeTok{y =} \StringTok{\textquotesingle{}\textquotesingle{}}\NormalTok{)}
\end{Highlighting}
\end{Shaded}

\begin{verbatim}
## `geom_smooth()` using method = 'gam' and formula = 'y ~ s(x, bs = "cs")'
\end{verbatim}

\begin{verbatim}
## Error in `geom_smooth()`:
## ! Problem while computing stat.
## i Error occurred in the 2nd layer.
## Caused by error in `compute_layer()`:
## ! `stat_smooth()` requires the following missing aesthetics: x and y.
\end{verbatim}

\begin{quote}
Write a few sentences here.
\end{quote}

\subsection{Question 7 {[}2 points{]}}\label{question-7-2-points}

\emph{Does this relationship change by whether the school is a research
university? Using the filter() function, create two versions of the
plot, one for research universities and the other for non-research
universities.}

\begin{Shaded}
\begin{Highlighting}[]
\NormalTok{df }\SpecialCharTok{\%\textgreater{}\%}
  \FunctionTok{filter}\NormalTok{() }\SpecialCharTok{\%\textgreater{}\%} \CommentTok{\# Filter to non{-}research universities}
  \FunctionTok{ggplot}\NormalTok{(}\FunctionTok{aes}\NormalTok{(}\AttributeTok{x =}\NormalTok{ , }\CommentTok{\# Put the explanatory variable on the x{-}axis}
             \AttributeTok{y =}\NormalTok{ )) }\SpecialCharTok{+}  \CommentTok{\# Put the outcome variable on the y{-}axis}
  \FunctionTok{geom\_point}\NormalTok{() }\SpecialCharTok{+} \CommentTok{\# Create a scatterplot}
  \FunctionTok{geom\_smooth}\NormalTok{() }\SpecialCharTok{+} \CommentTok{\# Add line of best fit}
  \FunctionTok{labs}\NormalTok{(}\AttributeTok{title =} \StringTok{\textquotesingle{}\textquotesingle{}}\NormalTok{, }\CommentTok{\# give the plot meaningful labels to help the viewer understand it}
       \AttributeTok{subtitle =} \StringTok{\textquotesingle{}\textquotesingle{}}\NormalTok{, }
       \AttributeTok{x =} \StringTok{\textquotesingle{}\textquotesingle{}}\NormalTok{,}
       \AttributeTok{y =} \StringTok{\textquotesingle{}\textquotesingle{}}\NormalTok{)}
\end{Highlighting}
\end{Shaded}

\begin{verbatim}
## `geom_smooth()` using method = 'gam' and formula = 'y ~ s(x, bs = "cs")'
\end{verbatim}

\begin{verbatim}
## Error in `geom_smooth()`:
## ! Problem while computing stat.
## i Error occurred in the 2nd layer.
## Caused by error in `compute_layer()`:
## ! `stat_smooth()` requires the following missing aesthetics: x and y.
\end{verbatim}

\begin{Shaded}
\begin{Highlighting}[]
\NormalTok{df }\SpecialCharTok{\%\textgreater{}\%}
  \FunctionTok{filter}\NormalTok{() }\SpecialCharTok{\%\textgreater{}\%} \CommentTok{\# Filter to research universities}
  \FunctionTok{ggplot}\NormalTok{(}\FunctionTok{aes}\NormalTok{(}\AttributeTok{x =}\NormalTok{ , }\CommentTok{\# Put the explanatory variable on the x{-}axis}
             \AttributeTok{y =}\NormalTok{ )) }\SpecialCharTok{+}  \CommentTok{\# Put the outcome variable on the y{-}axis}
  \FunctionTok{geom\_point}\NormalTok{() }\SpecialCharTok{+} \CommentTok{\# Create a scatterplot}
  \FunctionTok{geom\_smooth}\NormalTok{() }\SpecialCharTok{+} \CommentTok{\# Add line of best fit}
  \FunctionTok{labs}\NormalTok{(}\AttributeTok{title =} \StringTok{\textquotesingle{}\textquotesingle{}}\NormalTok{, }\CommentTok{\# give the plot meaningful labels to help the viewer understand it}
       \AttributeTok{subtitle =} \StringTok{\textquotesingle{}\textquotesingle{}}\NormalTok{, }
       \AttributeTok{x =} \StringTok{\textquotesingle{}\textquotesingle{}}\NormalTok{,}
       \AttributeTok{y =} \StringTok{\textquotesingle{}\textquotesingle{}}\NormalTok{)}
\end{Highlighting}
\end{Shaded}

\begin{verbatim}
## `geom_smooth()` using method = 'gam' and formula = 'y ~ s(x, bs = "cs")'
\end{verbatim}

\begin{verbatim}
## Error in `geom_smooth()`:
## ! Problem while computing stat.
## i Error occurred in the 2nd layer.
## Caused by error in `compute_layer()`:
## ! `stat_smooth()` requires the following missing aesthetics: x and y.
\end{verbatim}

\subsection{Question 8 {[}1 point{]}}\label{question-8-1-point}

\emph{Instead of creating two separate plots, color the points by
whether the school is a research university. To do this, you first need
to modify the research\_u variable to be categorical (it is currently
stored as numeric). To do this, use the mutate command with
\texttt{ifelse()} to create a new variable called
\texttt{research\_u\_cat} which is either ``Research'' if
\texttt{research\_u} is equal to 1, and ``Non-Research'' otherwise.}

\begin{Shaded}
\begin{Highlighting}[]
\NormalTok{df }\OtherTok{\textless{}{-}}\NormalTok{ df }\SpecialCharTok{\%\textgreater{}\%}
  \FunctionTok{mutate}\NormalTok{(}\AttributeTok{research\_u\_cat =} \FunctionTok{ifelse}\NormalTok{()) }\CommentTok{\# Create a labeled version of the research\_u variable}
\end{Highlighting}
\end{Shaded}

\begin{verbatim}
## Error in `mutate()`:
## i In argument: `research_u_cat = ifelse()`.
## Caused by error in `ifelse()`:
## ! argument "test" is missing, with no default
\end{verbatim}

\begin{Shaded}
\begin{Highlighting}[]
\NormalTok{df }\SpecialCharTok{\%\textgreater{}\%}
  \FunctionTok{ggplot}\NormalTok{(}\FunctionTok{aes}\NormalTok{(}\AttributeTok{x =}\NormalTok{ , }\CommentTok{\# Put the explanatory variable on the x{-}axis}
             \AttributeTok{y =}\NormalTok{ , }\CommentTok{\# Put the outcome variable on the y{-}axis}
             \AttributeTok{color =}\NormalTok{ )) }\SpecialCharTok{+} \CommentTok{\# Color the points by the new variable you created above}
  \FunctionTok{geom\_point}\NormalTok{() }\SpecialCharTok{+} \CommentTok{\# Create a scatterplot}
  \FunctionTok{geom\_smooth}\NormalTok{() }\SpecialCharTok{+} \CommentTok{\# Add line of best fit}
  \FunctionTok{labs}\NormalTok{(}\AttributeTok{title =} \StringTok{\textquotesingle{}\textquotesingle{}}\NormalTok{, }\CommentTok{\# give the plot meaningful labels to help the viewer understand it}
       \AttributeTok{x =} \StringTok{\textquotesingle{}\textquotesingle{}}\NormalTok{,}
       \AttributeTok{color =} \StringTok{\textquotesingle{}\textquotesingle{}}\NormalTok{,}
       \AttributeTok{y =} \StringTok{\textquotesingle{}\textquotesingle{}}\NormalTok{)}
\end{Highlighting}
\end{Shaded}

\begin{verbatim}
## `geom_smooth()` using method = 'gam' and formula = 'y ~ s(x, bs = "cs")'
\end{verbatim}

\begin{verbatim}
## Error in `geom_smooth()`:
## ! Problem while computing stat.
## i Error occurred in the 2nd layer.
## Caused by error in `compute_layer()`:
## ! `stat_smooth()` requires the following missing aesthetics: x and y.
\end{verbatim}

\subsection{Extra Credit {[}2 points{]}}\label{extra-credit-2-points}

\emph{Write a short paragraph discussing your findings. What do you
think is going on in these data?}

\begin{quote}
Write a few sentences here
\end{quote}

\end{document}
